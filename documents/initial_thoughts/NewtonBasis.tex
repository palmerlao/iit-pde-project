\documentclass[]{article}
%\setlength{\marginparwidth}{0.5in}
\usepackage{amsmath,amssymb,amsthm,graphicx}
\usepackage[author-year]{amsrefs}

\newcommand{\vc}{\boldsymbol{c}}
\newcommand{\vd}{\boldsymbol{d}}
\newcommand{\vf}{\boldsymbol{f}}
\newcommand{\vk}{\boldsymbol{k}}
\newcommand{\vl}{\boldsymbol{l}}
\newcommand{\mK}{\mathsf{K}}
\newcommand{\mL}{\mathsf{L}}
\newcommand{\mM}{\mathsf{M}}
\newcommand{\vp}{\boldsymbol{p}}
\newcommand{\mP}{\mathsf{P}}
\newcommand{\tu}{\tilde{u}}
\newcommand{\vu}{\boldsymbol{u}}
\newcommand{\vv}{\boldsymbol{v}}
\newcommand{\mV}{\mathsf{V}}
\DeclareMathOperator{\spann}{span}

\textwidth 6.5in
\hoffset -1in

\begin{document}

\title{Newton Basis Thoughts}
\author{John Erickson, Greg E. Fasshauer, Fred J. Hickernell, \\
Palmer Lao, and Tim McCollam}
\maketitle 

Let $K$ be a symmetric, positive definite kernel.  Let us write an expression for a function approximation in terms of the traditional basis:
\[
\tu(x) = \sum_{j=1}^n K(x,x_j) c_j =  \vk^T(x) \vc \approx u(x), \qquad \vk(x)=(K(x,x_j))_{j=1}^n.
\]
According to the Schaback paper, there is another basis $\{v_1, \ldots, v_n\}$ with $\vv(x)=(v_j(x))_{j=1}^n$ and 
\begin{equation*}
\spann\{v_1, \ldots, v_j\}=\spann\{K(\cdot,x_1), \ldots, K(\cdot, x_j)\}, \qquad v_j(x_j)=1, \qquad v_j(x_i)=0, \ i=0, \ldots, j-1.
\end{equation*}
Thus, so $\vv(x) = \mM \vk(x)$ for some appropriate lower triangular matrix $\mM$, which is independent of $x$ but dependent on $\{x_1, \ldots, x_n\}$.
If we have some values of the function, $u$, i.e., $\vu=(u(x_1), \ldots, u(x_n))^T$, then interpolation yields:
\begin{gather*}
\mK=(K(x_i,x_j))_{i,j=1}^n, \qquad \mV=(v_j(x_i))_{i,j=1}^n, \qquad \mV=\mK \mM^T \\
u(x) =  \vk^T(x) \vc =  \vv^T(x) \vd, \qquad \vc=\mK^{-1} \vu, \qquad \vd=\mV^{-1} \vu = \mM^{-T} \vc,
\end{gather*}
where $\mV$ is lower triangular.

Now let $L$ be some known linear operator, e.g., a differential operator.  If you would like to construct $Lu(x)$, given the $u(x_i)$ values, then
\begin{gather*}
Lu(x) \approx L \tu(x) = \vl^T(x) \vc = \vl^T(x) \mK^{-1} \vu = \vl^T(x) \mM^T \mV^{-1} \vu = \vp^T(x) \mV^{-1} \vu,\\
 \vl(x):=(l_j)_{j=1}^n:=(L(K(\cdot,x_j)))_{j=1}^n, \qquad \vp(x):= (p_j)_{j=1}^n:=(L(v_j)(x))_{j=1}^n = \mM\vl(x).
\end{gather*}
So $\mV^{-1}\vu$ should be stable and easy to solve.

Suppose that we want to solve a differential equation
\[
Lu(x) = f(x), \qquad u(x)=?,
\]
where $f$ is some known function and $L$ may now include the boundary conditions.  Using collocation requires
\[
L \tu(x_i) = f(x_i), \ i=1, \ldots, n,
\]
which leads to 
\[
\tu(x) =  \vk^T(x) \vc, \qquad \text{where }\mL \vc = \vf, \qquad \mL=(l_j(x_i))_{i,j=1}^n=(L(K(\cdot,x_j))(x_i))_{i,j=1}^n, \qquad \vf=(f(x_i))_{i=1}^n.
\]
If we have to invert $\mL$, it may be bad.  

Alternatively, we can write this as as 
\begin{gather*}
\tu(x) =  \vv^T(x) \vd, \qquad \text{where }\mL \mM^T \vd = \mP \vd= \vf, \\
\mL \mM^T = \mP:= (p_j(x_i))_{i,j=1}^n=(L(v_j)(x_i))_{i,j=1}^n,
\end{gather*}




\end{document}
